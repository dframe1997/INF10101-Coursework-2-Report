\documentclass[12pt,a4paper]{article}
%
\usepackage{graphicx}
\usepackage{helvet}

\usepackage{parskip}% http://ctan.org/pkg/parskip
\usepackage{fancyhdr}
\usepackage{titlesec}
\PassOptionsToPackage{hyphens}{url}\usepackage{hyperref}
\usepackage{apacite}

\pagestyle{fancyplain}
\fancyhf{}
\renewcommand{\headrulewidth}{0.5pt}
\renewcommand{\footrulewidth}{0.5pt}
\setlength{\headheight}{15pt}
\fancyhead[L]{}
\fancyhead[R]{}
\fancyfoot[R]{INF10101}
\fancyfoot[L]{40200819}
\fancyfoot[C]{\thepage}

\renewcommand{\familydefault}{\sfdefault}
\linespread{1.5}
% Used for displaying a sample figure. If possible, figure files should
% be included in EPS format.
%
% If you use the hyperref package, please uncomment the following line
% to display URLs in blue roman font according to Springer's eBook style:
% \renewcommand\UrlFont{\color{blue}\rmfamily}

%\makeatletter
%\renewcommand\subsection{\@startsection {subsection}{1}{2mm} % name, level, indent
%                               {3pt plus 2pt minus 1pt} % before skip
%                               {3pt plus 0pt} % after skip
%                               {\normalfont\bfseries}}
%\makeatother
%\makeatletter
%\renewcommand\section{\@startsection {section}{1}{0mm} % name, level, indent
%                               {4pt plus 2pt minus 1pt} % before skip
%                               {4pt plus 0pt} % after skip
%                               {\bfseries}}
%\makeatother



\begin{document}
%

\newcommand{\HRule}{\rule{\linewidth}{0.5mm}}

\begin{titlepage}
	\begin{center}

	\HRule \\[0.4cm]
    	{\Large \bfseries INF10101 - Security Coursework\par}
	\vspace{0.2cm}
	\HRule \\[1.5cm]

	
    	\vspace{1cm}
	\begin{minipage}{0.8\textwidth}
	\begin{center} \large
        David Frame - 40200819
        	
				
   	 \end{center}
    	\end{minipage}
	
    	\begin{minipage}{1\textwidth}
    	\begin{center} \large
        
		Computing Science
    	\end{center}
    	\end{minipage}
    	
    \vspace{2cm}
    \begin{minipage}{0.8\textwidth}
	\begin{center} \large
        \emph{Word Count: 0/2000(2200) words}
        	
				
   	 \end{center}
    	\end{minipage}
	
	

    	\vfill

    	% Bottom of the page
	\begin{minipage}{1\textwidth}
    	\begin{center} \large
		School of Computing
    	\end{center}
    	\end{minipage}
	
	\vspace{1cm}
    	{\large \today}


	\end{center}
\end{titlepage}

\section{Threat Context}
\subsection{The organisation, it's information assets and it's vulnerabilities}
Forthview Surgery is a large, busy medical practice situated in the ground floor of Tummel House (scenario, para 13). The GPs at the surgery are effectively partners rather than employees as they are self employed under contract to the NHS; they receive a share of the practice's profits and organise their own holidays (scenario, para 14). The surgery has access to a lot of information about its staff and patients that could be valuable to malicious people.

These information assets may include personal information about patients and staff, which could be illegally sold to advertisers or used to blackmail the patients about sensitive information such as their sexual health. If an attacker wanted to harm an important individual, such as a politician, the person's health details, such as substances that they are allergic to, could be a valuable data asset (scenario, para 18). 

If money was an objective, attackers may want to sell prescription drugs on the black market, so systems dealing with prescriptions would also be information assets. Similarly, the surgery's computing power could be an exploitable asset for cryptocurrency miners, also known as cryptojacking.

Vulnerabilities making these information assets accessible to malicious people could include social engineering to manipulate staff into giving away passwords, personal information or prescriptions. This could be targeted at the administration staff by, for example, sending a fake letter requesting a prescription (spear phishing). The GPs may also be at risk of social engineering if they use the vision system (scenario, para 22) in public, as they could accidentally expose patient information. 

Public access to the building could be a vulnerability, especially if a perpetrator was willing to pose as a patient. In addition, the GP2GP service could be intercepted to steal information by posing as a GP who needs the patient's data (scenario, para 17).

The emergency care system is accessible by various parties, making the data vulnerable to many potentially manipulable people, or disgruntled employees (scenario, para 18). Dr Holloway and Dr Wells struggle to keep their administration up to date, which may cause a vulnerability if they don't update their passwords regularly. 

As the surgery is a joint data controller with the Lennox and Gilchrist pharmacy (scenario, para 18), one could make the other vulnerable in the case of a breach.

The following table lists the information assets mentioned earlier with vulnerabilities that may be exploited to access them.

\begin{center}
\begin{tabular}{ |p{0.5\textwidth}|p{0.5\textwidth}| } 
 \hline
 \textbf{Information Asset} & \textbf{Vulnerability} \\ 
  \hline
 Patient and staff details & Social engineering and carelessness of staff\\ 
  \hline
 Computing Power & Ageing Windows 2003 server \\ 
 \hline
 Information about important individuals & GP2GP Interception \\ 
 \hline
 Prescription systems & Spear Phishing \\ 
 \hline
 Emergency care system & Social engineering and/or disgruntled employees \\ 
 \hline
 GP Login & Infrequent updating of passwords \\ 
 \hline
\end{tabular}
\caption{Table 1: Information assets and related vulnerabilities}
\end{center}

\subsection{Identifying threat actors}
Due to the highly sensitive data being processed by the surgery, there could be many potential threat actors with various motivations. This section will discuss three of them.

Disgruntled ex employees are a common threat for some companies, though in this case they would be the lowest threat of the three actors. They would be unlikely to take down vital systems due to the potential harm it could inflict on patients. Instead, they may steal IP or staff information as retaliation for poor treatment. This threat may be higher than it would normally as none of the staff are actual employees of the company, they are contractors, so they may have less of a relationship to the other staff members.

Script Kiddies are another cause for concern. Motivations vary on a person to person basis, so their threat level is in the middle - they may just want to cause a bit of trouble for their own entertainment, or they might try to take data for ransom. If they have sinister motivations, they would still have less resources than organised crime.

Organised Crime is the biggest threat to most companies, and particularly vital services such as the NHS. They may aim to make money, which can be highly damaging by itself, but it could be for political or terrorism reasons (cyber warfare), intentionally harming people. This type of attack was a huge issue for the NHS during the WannaCry crisis in May 2017.

\subsection{Goal for further analysis}
After reviewing the threat actors, it seems most appropriate to analyse organised crime as it poses the highest threat. Both money and information could be goals of this threat actor, but the author ultimately chose to analyse the threat of organised crime attacking for monetary gain as it has happened recently with WannaCry and was a huge disaster for the NHS.

\section{Modelling the threat}
As this threat is based on ransomware, the author researched various models for this type of attack. One result that came up was  aversion of the Cyber kill chain that was specific to ransomware, incredibly it is called the ransomware kill chain and it was developed by hornet security. It allows the pharmacy/hospital/whatever to anticipate the stages of attack that the criminals may follow. Also raises awareness. 

\section{Analysing the threat}
WannaCry was based on a vulnerability in older versions of Windows. As the surgery uses an 'aging windows 2003' server (scenario, para 20), it may be vulnerable to similar attacks.

This is especially interesting as any sort of threat can cause systems to go down, which has the side effect of putting lives at risk - even if the goal is only to make money.
\section{Note for management}
Vulnerability assessment allows them to protect their patients proactively by anticipating threats and putting countermeasures in place. 

Threat modelling raises awareness for the various vulnerabilities highlighted by showing them diagrammatically, free from jargon and other barriers of understanding. 

Risk management helps to mitigate issues by evaluating their likelihood in advance and having a reasonable threshold where security will prevent riskier activities and ideas from happening.

[Approx. one page]

\bibliographystyle{apacite}
\bibliography{Bibliography}

\end{document}
